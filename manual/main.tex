% ==== Document Class & Packages =====
\documentclass[12pt,hidelinks]{article}
\usepackage[many]{tcolorbox}
\usepackage{amsmath}
\usepackage{graphicx}
\usepackage{xcolor}
\usepackage[english]{babel}
\usepackage{fancyhdr}
\usepackage{fourier}% change to lmodern if fourier is no available
\usepackage{fancyref}
\usepackage{hyperref}
\usepackage{cleveref}
\usepackage{listings}
\usepackage{varwidth}
\usepackage{geometry}
\tcbuselibrary{documentation}
\geometry{
    a4paper,
    left=33mm,
    right=33mm,
    top=20mm
}

% ========= Path to images ============
%   - Direct the computer on the path 
%       to the folder containg the images
% =====================================
\graphicspath{{./images/}}
% ============= Macros ================
\newcommand{\fillin}{\underline{\hspace{.75in}}{\;}}
\setlength{\parindent}{0pt}
\addto{\captionsenglish}{\renewcommand*{\contentsname}{Table of Contents}}
\linespread{1.2}
% ======== Footers & Headers ==========
\cfoot{\thepage}
\chead{}\rhead{}\lhead{}
% =====================================
\renewcommand{\thesection}{\arabic{section}}
\newcommand\sectionnumfont{% font specification for the number
    \fontsize{380}{130}\color{myblueii}\selectfont}
\newcommand\sectionnamefont{% font specification for the name "PART"
    \normalfont\color{white}\scshape\small\bfseries }
% ============= Colors ================
% ----- Red -----
\definecolor{mordantred19}{rgb}{0.68, 0.05, 0.0}
% ----- Blue -----
\definecolor{st.patrick\'sblue}{rgb}{0.14, 0.16, 0.48}
\definecolor{teal}{rgb}{0.0, 0.5, 0.5}
\definecolor{beaublue}{rgb}{0.74, 0.83, 0.9}
\definecolor{mybluei}{RGB}{0,173,239}
\definecolor{myblueii}{RGB}{63,200,244}
\definecolor{myblueiii}{RGB}{199,234,253}
% ---- Yellow ----
\definecolor{blond}{rgb}{0.98, 0.94, 0.75}
\definecolor{cream}{rgb}{1.0, 0.99, 0.82}
% ----- Green ------
\definecolor{emerald}{rgb}{0.31, 0.78, 0.47}
\definecolor{darkspringgreen}{rgb}{0.09, 0.45, 0.27}
% ---- White -----
\definecolor{ghostwhite}{rgb}{0.97, 0.97, 1.0}
\definecolor{splashedwhite}{rgb}{1.0, 0.99, 1.0}
% ---- Grey -----
\definecolor{whitesmoke}{rgb}{0.96, 0.96, 0.96}
\definecolor{lightgray}{rgb}{0.92, 0.92, 0.92}
\definecolor{floralwhite}{rgb}{1.0, 0.98, 0.94}
% ========= Hyper Ref ===========
\hypersetup{
    colorlinks,
    linkcolor={red!50!black},
    citecolor={blue!50!black},
    urlcolor={blue!80!black}
}
\begin{document}
\begin{titlepage}
    \centering % Center everything on the title page
    \scshape % Use small caps for all text on the title page
    \vspace*{1.5\baselineskip} % White space at the top of the page
% ===================
%    Title Section     
% ===================
    \rule{13cm}{1.6pt}\vspace*{-\baselineskip}\vspace*{2pt} % Thick rule
    \rule{13cm}{0.4pt} % Thin rule
    
    \vspace{0.75\baselineskip} % Whitespace above the title
% ========== Title ===============    
    {    \Huge MPI3SNP \\    }
% ======================================
    \vspace{0.75\baselineskip} % Whitespace below the title
    \rule{13cm}{0.4pt}\vspace*{-\baselineskip}\vspace{3.2pt} % Thin rule
    \rule{13cm}{1.6pt} % Thick rule
    
    \vspace{1.75\baselineskip} % Whitespace after the title block
% =================
%    Information    
% =================
    {\large 
        Christian Ponte-Fernández \\ \vspace{1mm}
        Jorge González-Domínguez \\ \vspace{2.5mm}
        María J. Martín
    } \\
    \vfill
    \includegraphics[width=.35\textwidth]{logo} \\
    \vspace{1mm}
    Contact: \url{christian.ponte@udc.es}\\ 
\end{titlepage}
\clearpage
\thispagestyle{empty}
\mbox{}
\clearpage
%%%%%%%%%%%%%%%%%%%%%%%%%%%%%%%%%%%%%%%%%%%%%%%%%%%%%%%%%%%
\setcounter{page}{1}
\tableofcontents
\newpage
\newgeometry{
    left=29mm, 
    right=29mm, 
    top=20mm, 
    bottom=15mm}
%%%%%%%%%%%%%%%%%%%%%%%%%%%%%%%%%%%%%%%%%%%%%%%%%%%%%%%%%%%
\section{What is MPI3SNP?}
    MPI3SNP is a parallel software tool dedicated to genome-wide association 
    studies, performing a third-order exhaustive search. It is targeted to 
    cluster architectures, and mitigates the cubic time complexity inherent to 
    third-order searches by exploiting the several layers of parallelism 
    present in a supercomputer. CPU and GPU implementations are offered.
%%%%%%%%%%%%%%%%%%%%%%%%%%%%%%%%%%%%%%%%%%%%%%%%%%%%%%%%%%%
\section{Building}
    Support is currently limited to linux distributions only.

    \subsection{Requirements}
        \begin{itemize}
            \item CMake (>3.0 version).
            \item A C++14 compatible compiler.
            \item MPI library.
            \item CUDA (optional).
        \end{itemize}
        
    \subsection{Compilation}
        CMake is the project build manager. CMake should be able to determine 
        installed compilers and libraries. If this is is not the case, please 
        refer to your CMake version's documentation. By default, CMake will 
        check for a CUDA installation and set the target architecture 
        accordingly. This behaviour can be manually controlled by setting the 
        \linebreak \texttt{TARGET\_ARCH} CMake variable to \texttt{CPU} or 
        \texttt{GPU}.

        Building the sources looks like this:
        \begin{verbatim}
            cd MPI3SNP/project/path
            mkdir build
            cd build/
            cmake ..
            make -j4
        \end{verbatim}
%%%%%%%%%%%%%%%%%%%%%%%%%%%%%%%%%%%%%%%%%%%%%%%%%%%%%%%%%%%
\clearpage
\section{Usage}
    MPI3SNP takes two files as the input, using the PLINK/TPED format, and 
    writes the results to a third file. All file paths are provided to the 
    program as positional arguments as follows:
    
    \begin{verbatim}
        ./MPI3SNP <path/to/tped> <path/to/tfam> <path/to/output>
    \end{verbatim}

    Additional configuration options (specific to either CPU or GPU 
    implementation) are available to the user, and can be consulted using the 
    \texttt{-h} flag.

    \subsection{Sample files}
        Sample files can be found on MPI3SNP's wiki \cite{wiki}. These are a 
        syntetic dataset used for performance evaluation, which describe the 
        input file format and can be used for verification/evaluation purposes.
%%%%%%%%%%%%%%%%%%%%%%%%%%%%%%%%%%%%%%%%%%%%%%%%%%%%%%%%%%%
\section{Troubleshooting}
    Support is currently limited to linux distributions only. If you are having 
    trouble building/using the application, please submit a new issue to get 
    help.
%%%%%%%%%%%%%%%%%%%%%%%%%%%%%%%%%%%%%%%%%%%%%%%%%%%%%%%%%%%
\section{License}
    This software is licensed under the GPU GPLv3 license. Check the license 
    file \cite{license} for details.
%%%%%%%%%%%%%%%%%%%%%%%%%%%%%%%%%%%%%%%%%%%%%%%
\newpage
\begin{thebibliography}{2}
    \bibitem{wiki}
        Sample files from MPI3SNP's wiki.\\
        \url{https://github.com/chponte/mpi3snp/wiki/Sample-files}
    \bibitem{license}
        License file included with MPI3SNP.\\
        \url{https://github.com/chponte/mpi3snp/blob/master/LICENSE.md}
\end{thebibliography}
\addtocounter{section}{1}
\addcontentsline{toc}{section}{\protect\numberline{\thesection}References}
%%%%%%%%%%%%%%%%%%%%%%%%%%%%%%%%%%%%%%%%%%%%%%%%%%%%%%%%%%%%%%%%%%
\end{document}